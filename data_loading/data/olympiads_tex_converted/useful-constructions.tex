\documentclass[10pt]{article}

\usepackage{graphics,epsfig,calc,multicol}
\usepackage{amsmath,amscd,amssymb}
\usepackage{pstricks}
\usepackage{pst-plot}
\usepackage{pstricks-add}
\usepackage{epsfig}
\usepackage{graphicx,color}
\usepackage{amssymb,amsthm,amsfonts,amsmath,pifont}

\newtheorem{fact}{Fact}

\theoremstyle{definition}
\newtheorem{example}{Example}

\theoremstyle{remark}
\newtheorem*{sol}{Solution}

\psset{xunit=1.0cm,yunit=1.0cm,algebraic=true,dotstyle=o,dotsize=3pt 0,linewidth=0.8pt,arrowsize=3pt 2,arrowinset=0.25}


\global\oddsidemargin=0pt
\global\evensidemargin=0pt
\global\topmargin=-20pt
%\global\headheight=35pt
\global\headsep=35pt
\global\textheight=642pt
\global\textwidth=452pt


\begin{document}
\title{Some useful constructions}
\author{Carlos Shine}
\date{}
\maketitle
\begin{abstract}
Suppose you have a geometry problem and some really weird relation between sides and/or angles. What do you do then? Trig-bash? Complex-bash? Bash? You \textbf{can} do that, of course, but you can save a lot of computations just by making a simple construction.
\end{abstract}

\section{Some tricks}
\begin{itemize}
\item \textbf{Midpoints and angle chasing.} Midpoints can be really annoying in a geometry problem, specially if mixed with a lot of angles. How do you relate midpoints with angles? There are at least two possibilities:
\begin{itemize}
\item \textbf{Midlines.} You should know that if $M$ and $N$ are the midpoints of segments $AB$ and $AC$ then $BC$ is parallel to $MN$. So you might want to construct another midpoint!
\item \textbf{Paralellograms.} In paralellograms, the diagonals meet at their respective midpoints. You might want to take advantage of that.
\item Another nice ratio is $2:1$: it is the centroid ratio. So if a point $X$ divides $AB$ in a $1:2$ ratio, you can try to construct a triangle with centroid $X$ and median $AB$.
\end{itemize}

\item \textbf{Divide and conquer.} You can divide some segments and/or angles in parts, especially when you have relations like ``this angle is three times that other'' or $AB = AC + DE$.
\item \textbf{More on segments.} If a broken line has the same length as a segment or another broken line, a good idea is to straighten all broken lines. This usually generates a lot of isosceles triangles, and you know good things happen when isosceles triangles are around.
\item \textbf{Circles with radius $0$.} Sometimes it's handy to consider a point a circle with radius $0$. Especially if you apply power of a point.
\item \textbf{Segment ratios and angles.} Suppose you have some segment ratios, but still have to deal with angles. One good bridge between these two apparently disparate worlds is the Apolonius circle: let $A$ and $B$ be two distinct fixed points; the locus of the points $X$ such that $\frac{AX}{BX}$ is fixed and different from $1$ is a circle with $AB$ as an axis of symmetry.
\item \textbf{Point redefinition.} You want a point to have a property, but can't get a proof of it! So try to construct a point with the same property and prove that it is actually the same point.
\item \textbf{Projecting a point onto the sides of a polygon.} I know it sounds a bit random, but projecting a point onto the sides of a polygon generates a lot of cyclic quads, and some nice angles may appear.
\end{itemize}

\section{Problems}
\begin{enumerate}
% IMO95-like geo inequality
\item (Peru Cono Sur TST 2008) Let $ABCDEF$ be a convex hexagon such that $\angle FAB = \angle CDE = 90^\circ$ and the quadrilateral $BCEF$ has an inscribed circle. Prove that $AD \leq BC + FE$.

% Embedding into 3D
\item (Iberoamerican 2003) In a square $ABCD$, let $P$ and $Q$ be points on the sides $BC$ and $CD$ respectively, different from its endpoints, such that $BP = CQ$. Consider points $X$ and $Y$ such that $X\neq Y$, in the segments $AP$ and $AQ$ respectively. Show that, for every $X$ and $Y$ chosen, there exists a triangle whose sides have lengths $BX$, $XY$ and $DY$.

\item Prove the \emph{Erd\H os-Mordell inequality}: Let $P$ be a point inside triangle $ABC$ and $d_a$, $d_b$, $d_c$ be the distances from $P$ to sides $BC$, $CA$, $AB$ respectively. Then $PA + PB + PC \geq 2(d_a + d_b + d_c)$.

% Nice construction, midpoints
\item (Iberoamerican Shortlist 2010, Brazil TST 2011) Two circles $C_1$ and $C_2$, with centers $O_1$ and $O_2$ respectively, intersect at points $A$ and $B$. Let $X$ and $Y$ be points on $C_1$ distinct from $A$ and $B$. Lines $XA$ and $YA$ meet $C_2$ again at $Z$ and $W$, respectively.

Let $M$ be the midpoint of $O_1O_2$, $S$ be the midpoint of $XA$ and $T$ be the midpoint of $WA$. Prove that $MS = MT$ if and only if $X$, $Y$, $Z$ and $W$ lie on the same circle.

\item (IMO 1996) Let $ABCDEF$ be a convex hexagon such that $AB$ is parallel to $DE$, $BC$ is parallel to $EF$, and $CD$ is parallel to $FA$. Let $R_A$, $R_C$, $R_E$ denote the circumradii of triangles $FAB$, $BCD$, $DEF$, respectively, and let $P$ denote the perimeter of the hexagon. Prove that 
$$R_{A}+R_{C}+R_{E}\geq\frac{P}{2}.$$	

\item (IMO 1995) Let $ABCDEF$ be a convex hexagon with $AB = BC = CD$ and $DE = EF = FA$, such that $\angle BCD =\angle EFA =\frac{\pi}3$. Suppose $G$ and $H$ are points in the interior of the hexagon such that $\angle AGB =\angle DHE =\frac{2\pi}3$. Prove that $AG+GB+GH+DH+HE\geq CF$.

\item (Brazil 2006) Let $ABC$ be a triangle. The internal bisector of $\angle B$ meets $AC$ in $P$. Let $I$ be the incenter of $ABC$. Prove that if $AP+AB = CB$, then $API$ is an isosceles triangle.
 
\item (IMO 1997) It is known that $\angle BAC$ is the smallest angle in the triangle $ABC$. The points $B$ and $C$ divide the circumcircle of the triangle into two arcs. Let $U$ be an interior point of the arc between $B$ and $C$ which does not contain $A$. The perpendicular bisectors of $AB$ and $AC$ meet the line $AU$ at $V$ and $W$, respectively. The lines $BV$ and $CW$ meet at $T$.

Show that $AU = TB + TC$.

\item (IMO 2001) Let $ABC$ be a triangle with $\angle BAC = 60^\circ$. Let $AP$ bisect $\angle BAC$ and let $BQ$ bisect $\angle ABC$, with $P$ on $BC$ and $Q$ on $AC$. If $AB + BP = AQ + QB$, what are the angles of the triangle? 

\item (Brazil 2001) In a convex quadrilateral, the {\sl altitude} relative to a side is defined to be the line perpendicular to this side through the midpoint of the opposite side. Prove that the four altitudes have a common point if and only if the quadrilateral is cyclic, that is, if and only if, there exists a circle which contains its four vertices.

\item (IMO 2007) Consider five points $A$, $B$, $C$, $D$ and $E$ such that $ABCD$ is a parallelogram and $BCED$ is a cyclic quadrilateral. Let $\ell$ be a line passing through $A$. Suppose that $\ell$ intersects the interior of the segment $DC$ at $F$ and intersects line $BC$ at $G$. Suppose also that $EF = FG = EC$. Prove that $\ell$ is the bisector of angle $\angle DAB$.

\item (IMOSL 2007) Denote by $M$ the midpoint of side $BC$ in an isosceles triangle $ABC$ with $AC = AB$. Take a point $X$ on the smaller arc $MA$ of circumcircle of triangle $ABM$. Denote by $T$ the point inside of angle $\angle BMA$ such that $\angle TMX = 90^\circ$ and $TX = BX$.

Prove that $\angle MTB - \angle CTM$ does not depend on choice of $X$.

\item Triangle $ABC$ is such that $AB = AC$. Let $D$ be a point on side $BC$ such that $BD = 2DC$. Point $P$ lies on segment $AD$ and satisfies $\angle ABP = \angle PAC$. Prove that $\angle BAC = 2\angle DPC$.

\item (Bosnia and Herzegovina 2011) Let $ABC$ be a triangle such that $AB + AC = 2BC$. Prove that the midpoints $M$ of $AB$ and $N$ of $AC$, the incenter $I$ of $ABC$ and $A$ lie on the same circle.

\item (Iran 2011) Let $ABC$ be a triangle. Line $r$ intersects the extension of $AB$ at $D$ ($B$ between $A$ and $D$) and the extension of $AC$ in $E$ ($C$ between $A$ and $E$). Suppose that reflection of line $\ell$ with respect to the perpendicular bisector of side $BC$ intersects the mentioned extensions in $D'$ and $E'$ respectively. Prove that if $BD + CE = DE$, then $BD' + CE' = D'E'$.

\item Let $A$ be a point in the interior of triangle $BCD$ such that $AB\cdot CD = AD\cdot BC$. Point $P$ is the symmetrical of point $A$ with respect to $BD$. Prove that $\angle PCB=\angle ACD$.

\item (USAMO 2005) Let $ABC$ be an acute-angled triangle, and let $P$ and $Q$ be two points on its side $BC$. Construct a point $C_1$ in such a way that the convex quadrilateral $APBC_1$ is cyclic, $QC_1\parallel CA$, and the points $C_1$ and $Q$ lie on opposite sides of the line $AB$. Construct a point $B_1$ in such a way that the convex quadrilateral $APCB_1$ is cyclic, $QB_1\parallel BA$, and the points $B_1$ and $Q$ lie on opposite sides of the line $AC$.

Prove that the points $B_1$, $C_1$, $P$, and $Q$ lie on a circle.

\item (USAMO 2009) Trapezoid $ABC$, with $AB \parallel CD$, is inscribed in circle $\omega$ and point $G$ lies inside triangle $BCD$. Rays $AG$ and  $BG$ meet $\omega$ again at points $P$ and $Q$, respectively. Let the line through $G$ parallel to $AB$ intersect $BD$ and $BC$ at points $R$ and $S$, respectively. Prove that quadrilateral $PQRS$ is cyclic if and only if $BG$ bisects $\angle CBD$.

\item (Iran TST 2011) In acute triangle $ABC$, $\angle B > \angle C$. Let $M$ be the midpoint of $BC$, $D$ and $E$ be the feet of the altitudes from $C$ and $B$ respectively. $K$ and $L$ are midpoints of $ME$ and $MD$ respectively. If $KL$ intersects the line through $A$ parallel to $BC$ at $T$, prove that $TA = TM$.

\item (USAMO 1999) Let $ABCD$ be an isosceles trapezoid with $AB \parallel CD$. The inscribed circle $\omega$ of triangle $BCD$ meets $CD$ at $E$. Let $F$ be a point on the (internal) angle bisector of $\angle DAC$ such that $EF \perp CD$. Let the circumscribed circle of triangle $ACF$ meet line $CD$ at $C$ and $G$. Prove that the triangle $AFG$ is isosceles.

\item (USA TST 2000) $ABCD$ is a cyclic quadrilateral. The projections of the intersection of its diagonals to sides $AB$ and $CD$ are $E$, $F$ respectively. Show that the line $EF$ is perpendicular to the line containing the midpoints of the sides of $BC$ and $DA$.

\item (USA TST 2001) In triangle $ABC$, $\angle B = 2\angle C$. Let $P$ and $Q$ be points on the perpendicular bisector of segment $BC$ such that rays $AP$ and $AQ$ trisect $\angle BAC$. Prove that $PQ < AB$ if and only if $\angle ABC$ is obtuse.

\item (IMOSL 2009) Let $ABC$ be a triangle. The incircle of $ABC$ touches the sides $AB$ and $AC$ at the points $Z$ and $Y$, respectively. Let $G$ be the point where the lines $BY$ and $CZ$ meet, and let $R$ and $S$ be points such that the two quadrilaterals $BCYR$ and $BCSZ$ are parallelograms.

Prove that $GR = GS$.

\item (IMOSL 2009) Given a cyclic quadrilateral $ABCD$, let the diagonals $AC$ and $BD$ meet at $E$ and the lines $AD$ and $BC$ meet at $F$. The midpoints of $AB$ and $CD$ are $G$ and $H$, respectively. Show that $EF$ is tangent at $E$ to the circle through the points $E$, $G$ and $H$.

\item (IMOSL 2008) Let $ABCD$ be a convex quadrilateral. Prove that there exists a point $P$ inside the quadrilateral such that 
$$\angle PAB+\angle PDC =\angle PBC+\angle PAD =\angle PCD+\angle PBA =\angle PDA+\angle PCB = 90^{\circ}$$
if and only if the diagonals $AC$ and $BD$ are perpendicular.

\item (IMOSL 2010) Let $ABCDE$ be a convex pentagon such that $BC\parallel AE$, $AB = BC + AE$ and $\angle ABC = \angle CDE$. Let $M$ be the midpoint of $CE$ and let $O$ be the circumcenter of triangle $BCD$. Given that $\angle DMO = 90^\circ$, prove that $2\angle BDA = \angle CDE$.

\end{enumerate}
\end{document}