\documentclass[11pt]{article}

\usepackage{amsmath}
\usepackage{amsthm}
\usepackage{amsfonts}
\usepackage{arcs}
\usepackage{wasysym}
\usepackage{tikz}
\usepackage{maria}

\newtheorem{lemma}{Lemma}
\theoremstyle{definition}
\newtheorem*{example}{Example}
\newtheorem*{exercise}{Exercise}

\newtheorem*{definition}{Definition}
\topmargin -0.5in
\oddsidemargin 0in
\textwidth 6in
\textheight 9.25in


\title{Generating Functions}

\author{Maria Monks -- Red Group}
\date{June 19, 2012}

\pagestyle{empty}

\begin{document}

\maketitle{}

\begin{quote}
  A generating function is a clothesline on which we hang a sequence of numbers up for display.
\end{quote}
\hfill --Herbert Wilf, \textit{Generatingfunctionology}

\section*{Generating function basics}

\indent  Generating functions are a useful tool for solving combinatorial problems.

A classic example of a generating function identity is the geometric series formula $$\frac{1}{1-x}=1+x+x^2+x^3+\cdots.$$  (The term \textit{function} is misleading - here $x$ is just a formal symbol, and the coefficients of the series are the important part!)

\begin{definition}
The \textit{(ordinary) generating function} of the sequence $c_0,c_1,c_2,\ldots$ with variable $x$ is the expression $$c_0+c_1x+c_2x^2+\cdots.$$  We abbreviate this series as $$\sum_{i=0}^\infty c_ix^i.$$
\end{definition}

Generating functions can be added and multiplied together.  They can also be differentiated, and sometimes composed!  

The following are \textit{definitions} of addition, multiplication, differentiation, and composition of generating functions (we're starting from the beginning here - no calculus allowed.)

\begin{itemize}
 \item \textbf{Addition:} $\sum_{i=0}^\infty a_i x^i+\sum_{i=0}^\infty b_i x^i=\sum_{i=0}^\infty (a_i+b_i) x^i$
 \item \textbf{Multiplication:} $\left(\sum_{i=0}^\infty a_i x^i\right)\cdot \left(\sum_{i=0}^\infty a_i x^i\right)=\sum_{n=0}^\infty \left(\sum_{i=0}^n a_ib_{n-i}\right)x^n$
 \item \textbf{Differentiation:} $\frac{d}{dx} \left(\sum_{n=0}^\infty a_n x^n\right)=\sum_{i=1}^\infty na_nx^{n-1}$
 \item \textbf{Composition:} If $F(x)=f_0+f_1x+f_2x^2+\cdots$ and $G(x)=g_1x+g_2x^2+\cdots$, then $$F\circ G(x)=\sum_{n\ge 0} f_nG(x)^n=\sum_{N=0}^\infty h_N x^N$$ where $$h_N=\sum_{s_1+\cdots+s_k=N} f_kg_{s_1}g_{s_2}\cdots g_{s_n}.$$
\end{itemize}

\begin{exercise}
  Is there a generating function that behaves like an additive identity?  A multiplicative identity?  Can subtraction and division of generating functions be defined?  Why did we only define composition above in the case that $G$ has no constant term?
\end{exercise}

\begin{exercise}
  Use the definitions above to prove the generating function identity $$\frac{1}{1-x}=1+x+x^2+x^3+\cdots.$$  Notice that $1-x=1-x+0\cdot x^2+0\cdot x^3+\cdots$ is a generating function as well.
\end{exercise}

\section*{Tricks for manipulating generating functions}

Let $F(x)=\sum_{n=0}^\infty a_n x^n$ and $G(x)=\sum_{n=0}^\infty b_n x^n$ be generating functions over $x$.  Try your hand at proving the following identities, using only the definitions above.

\begin{itemize}
  \item $x F(x)=\displaystyle\sum_{n=1}^\infty a_{n-1}x^n$
  \item $\displaystyle{\frac{F(x)-a_0}{x} = \sum_{n=0}^\infty a_{n+1}x^n}$
  \item $\displaystyle{\frac{d}{dx} (F(x)+G(x))=\frac{d}{dx}F(x)+\frac{d}{dx}G(x)}$
  \item $\displaystyle{\frac{d}{dx} (F(x)G(x))=G(x)\cdot\frac{d}{dx}F(x)+F(x)\cdot \frac{d}{dx}G(x)}$
  \item If $b_0\neq 0$, $G(x)$ has a multiplicative inverse: $$G(x)^{-1}=b_0^{-1}-b_0^{-1}b_1x+ (b_0^{-3}b_1^2-b_0^{-2}b_2)x^2+\cdots.$$
  \item If $b_0\neq 0$, $\displaystyle{\frac{d}{dx} \left(\frac{F(x)}{G(x)}\right)=\frac{G(x)\frac{d}{dx}F(x)-F(x)\frac{d}{dx}G(x)}{G(x)^2}}$.
\end{itemize}

\begin{exercise}  Show that
 $$\frac{1}{(1-x)^2}=1+2x+3x^2+4x^3+5x^4+\cdots.$$ 
\end{exercise}

\section*{Using generating functions to solve recurrences}

Suppose we wish to find an explicit formula for the $n$th Fibonnacci number $F_n$, where $F_0=0$, $F_1=1$, and $F_{n}=F_{n-1}+F_{n-2}$ for all $n\ge 2$.  Consider the generating function $$G(x)=\sum_{n=0}^\infty F_n x^n.$$  Let's manipulate this to take advantage of the recursion:
$$ G(x)-xG(x)-x^2G(x) = F_0+F_1x-F_0x+\sum_{n=2}^\infty (F_n-F_{n-1}-F_{n-2}) x^n =x.$$
Thus $G(x)=x/(1-x-x^2)$.  Using partial fractions and expanding each term as a geometric series, we find that $$G(x)=\sum_{n=0}^\infty \frac{1}{\sqrt{5}}\left(\left(\frac{1+\sqrt{5}}{2}\right)^n-\left(\frac{1-\sqrt{5}}{2}\right)^n\right)x^n,$$ and so $F_n=\frac{1}{\sqrt{5}}\left(\left(\frac{1+\sqrt{5}}{2}\right)^n-\left(\frac{1-\sqrt{5}}{2}\right)^n\right)$.

\begin{exercise}
  Use generating functions to find an explicit formula for the $n$th term in the sequence $a_n$ where $a_0 = 1$, $a_1 = 5$, $a_{n+2} = 4a_{n+1} -3a_n$.
\end{exercise}

\section*{Exponential generating functions}

We now have a good handle on ordinary generating functions.  But there are other useful generating functions associated to a given sequence of numbers!

\begin{definition} The \textit{exponential generating function} for the sequence $\{a_i\}_{i=0}^\infty$ is the series $\sum_{n=0}^\infty \frac{a_n}{n!}x^n$.  \end{definition}

\textbf{When to use them:}  Exponential generating functions are often useful for dealing with sequences that count \textit{labeled} objects, whereas ordinary generating functions are better for \textit{unlabeled} objects.  

For instance, if we let $p(n)$ be the number of ways of sorting $n$ indistinguishable balls into various groups, and let $B(n)$ be the number of ways of sorting $n$ balls labeled $1,2,\ldots,n$ into various groups, it is easy to work with the ordinary generating function for $p(n)$ and the exponential generating function for $B(n)$, but not the other way around. (See Problems.)

\textbf{Behaviour:}  The product of exponential generating functions behaves somewhat differently from that of ordinary generating functions:
$$\left(\sum_{n=0}^\infty \frac{a_n}{n!}x^n\right)\cdot \left(\sum_{n=0}^\infty \frac{b_n}{n!}x^n\right)=\sum_{n=0}^\infty \frac{1}{n!}\left(\sum_{k=0}^n\binom{n}{k}a_k b_{n-k}\right)x^n$$

\begin{exercise}
What do addition and differentiation do to exponential generating functions?
\end{exercise}

We can define $e^{x}$, $\sin(x)$, and $\cos(x)$ to be the exponential generating functions shown below.  
\begin{itemize}
  \item $e^{x}=\sum_{n=0}^\infty \frac{1}{n!}x^n$
  \item $\sin(x)=\sum_{n=0}^\infty (-1)^n\frac{1}{(2n+1)!}x^{2n+1}$
  \item $\cos(x)=\sum_{n=0}^\infty (-1)^n\frac{1}{(2n)!}x^{2n}$
\end{itemize}

\begin{exercise}
  Using the definition of $e^x$ as a generating function, show that $e^{x}e^{y}=e^{x+y}$ and that $\frac{d}{dx}e^{x}=e^x$.
\end{exercise}

And now for some...

\section*{Problems!}

\begin{enumerate}
\item Prove the following generating function identities:
   \begin{enumerate}
     \item $(1+x)^{n}=\sum_{k=0}^n \binom{n}{k}x^k$
     \item $(1-x)^{-n}=\sum_{k=0}^\infty \binom{-n}{k}(-x)^k=\sum_{k=0}^\infty \binom{n+k-1}{k} x^k$
     \item $\frac{1}{1-y(1+x)}=\sum_{k,n}\binom{n}{k}x^ky^n$
   \end{enumerate}

\item (Andy Niedermaier.)  Find a closed form for the generating function for each of the following sequences, and use it to find an explicit formula for $a_n$:
\begin{enumerate}
  \item $a_0 = 1$, $a_1 = 6$, $a_{n+2} = 4a_{n+1} - 4a_n$
  \item $a_0 = 0$, $a_1 = 5$, $a_2 = 47$, $a_{n+3} = 31a_{n+1} + 30a_n$
  \item $a_0 = a_1 = 1$, $a_{n+2} = a_{n+1} + 6a_n + n$
\end{enumerate}

\item Simplify $\sum_{n=0}^\infty n^2 x^n$.  (There are at least three nice ways of doing this!)

\item Prove the following combinatorial identities using generating functions:
\begin{enumerate}
   \item $\sum_{k=0}^n\binom{n}{k}^2=\binom{2n}{n}$
   \item $\sum_{k=0}^n k \binom{n}{k}=n\cdot 2^{n-1}$
   \item $\binom{n}{0}+\binom{n}{2}+\binom{n}{4}+\cdots=\binom{n}{1}+\binom{n}{3}+\binom{n}{5}+\cdots=2^{n-1}$
\end{enumerate}

\item Find an explicit formula for $$\binom{n}{0}+\binom{n}{3}+\binom{n}{6}+\binom{n}{9}+\cdots$$ in terms of $n$.

\item  (360 Problems for Mathematical Contests.) Find a closed form expression for the sum $$S_n=\binom{n}{1}-3\binom{n}{3}+5\binom{n}{5}-7\binom{n}{7}+\cdots.$$

\item In each of the following, find the ordinary generating function for the sequence whose $n$th term is described below.
   \begin{enumerate}
     \item The number of subsets of $\{1,2,\ldots,n\}$.
     \item $p(n)$, the number of \textit{partitions} of $n$.  That is, the number of ways of writing $n$ as a sum of positive integers where order doesn't matter.
     \item The number of partitions of $n$ into distinct parts.
     \item The number of partitions of $n$ into odd parts.  How does this relate to the generating function for partitions into distinct parts?
     \item The number of \textit{$k$-compositions} of $n$, that is, ways of writing $n$ as a sum of $k$ \textit{ordered} nonnegative integers $s_1,\ldots,s_k$.
     \item The number of integer lattice paths from $(0,0)$ to $(n,n)$, going either one unit right or one unit up at each step, such that every point $(x,y)$ on the path satisfies $y\le x$.
   \end{enumerate}

\item (HMMT 2007.)  Let $S$ denote the set of all triples $(i,j,k)$ of positive integers satisfying $i+j+k=17$.  Compute $$\sum_{(i,j,k)\in S} ijk.$$

\item \textbf{Signless Stirling Numbers of the First Kind:} Let $c(n,k)$ be the number of permutations of $\{1,2,\ldots,n\}$ which are a product of $k$ cycles.  Prove that $$\sum_{k}c(n,k)y^k=y(y+1)\cdots(y+n-1).$$

\item \textbf{Stirling Numbers of the Second Kind:} Let $S(n,k)$ be the number of ways of partitioning a set of $n$ labeled elements into $k$ nonempty unordered blocks. Show that $$\sum_{n} S(n,k)x^n=\frac{x^k}{(1-x)(1-2x)\cdots (1-kx)}.$$

\item \textbf{Derangements:} Let $D_n$ be the number of \textit{derangements} of $n$, that is, the number of permutations $\phi$ of $\{1,2,\ldots,n\}$ such that $\phi(i)\neq i$ for any $1\le i \le n$.  Find a closed form expression for the exponential generating function of $D_n$.  (Hint: Multiply $D(x)=\sum_{n\ge 0} \frac{D_n}{n!} x^n$ by $e^x$.)

\end{enumerate}

\end{document}
