\documentclass[12pt]{article}

\input{preamble}

\begin{document}

\pagestyle{fancy}
\setlength{\headheight}{15.2pt}

% Above the line stuff.

\lhead{\textit{MOSP 2011}}
\chead{\textit{Black}}
\rhead{\textit{Joshua Nichols-Barrer}}

% Title here.

{\bfseries \Large Stuff $\pmod p$}

% Get to work.

\begin{enumerate}

\item Let $p=4k+3$ be a prime number.  Find the number of different
  residues modulo $p$ of $(x^2+y^2)^2$, where $\gcd (x,p)=\gcd(y,p)=1$.

\item Let both $p$ and $2p+1$ be primes.  There are a total of $2p+1$
  balls in two boxes, and neither is empty.  Each step, one is allowed
  to move exactly half the balls in one box to the other box.  Let $k$
  be any positive integer less than $2p+1$.  Prove that there is a
  stage such that there are exactly $k$ balls in one of the boxes.
\item Let $p$ be an odd prime and let \[f(x) =
  \sum_{i=1}^{p-1}\left(\frac
  ip\right)x^{i-1}.\]  \begin{enumerate} \item Prove that $f$ is
    divisible by $x-1$ but not by $(x-1)^2$ if and only if $p\equiv
    3\pmod 4$. \item Prove that if $p\equiv 5\pmod 8$ then $f$ is
    divisible by $(x-1)^2$ and not by $(x-1)^3$.
\end{enumerate}

\item Find all odd primes $p$ such that both of the
  numbers \[n_1=1+p+p^2+\cdots+p^{p-2}+p^{p-1}\textrm{ and }
  n_2=1-p+p^2-\cdots-p^{p-2}+p^{p-1}\] are primes.

\item Find all surjective functions $f:\mathbb N\rightarrow\mathbb N$
  such that for every $m,n\in\mathbb N$ and every prime $p$, the
  number $f(m+n)$ is divisible by $p$ if and only if $f(m)+f(n)$ is
  divisible by $p$.  (Here $\mathbb N$ denotes the set of all positive integers.)

\item Let $p$ be a prime such that $p=k\cdot 2^n+1$, where $k$ is odd,
  $k>1$.  Suppose that $p$ divides Fermat number $2^{2^m}+1$ for some
  integer $m$ with $m\leq n-2$.  Prove that $k^{2^{n-1}}$ is congruent
    to 1 modulo $p$.

\item Let $\{x_n\}_{n=1}^\infty$ be a sequence with $x_1=2$, $x_2=12$,
  and $x_{n+2}=6x_{n+1}-x_n$ for every positive integer $n$.  Let $p$
  be an odd prime, and let $q$ be a prime divisor of $x_p$.  Prove
  that if $q>3$, then $q\geq 2p-1$.

\item Let $p$ be a prime, and let $k$ be an integer greater than 2.
  There are integers $a_1,a_2,\ldots,a_k$ such that $p$ divides
  neither $a_i$ ($1\leq i\leq k$) nor $a_i-a_j$ ($1\leq i<j\leq k$).
  Denote by $S$ the set \[\{n | 1\leq n\leq
  p-1,(na_1)_p<(na_2)_p<\cdots<(na_k)_p\},\] where $(b)_p$ denotes
  the remainder when $b$ is divided by $p$.  Prove that $S$ contains
  less than $\frac {2p}{k+1}$ elements.

\item Let $p>2$ be a prime number, and let $S=\{0,1,\ldots,p-1\}$.
  Determine the number of 6-tuples $(x_1,x_2,x_3,x_4,x_5,x_6)$ with
  $x_i\in S, 1\leq i\leq 6$, such that \[x_1^2+x_2^2+x_3^2\equiv
  x_4^2+x_5^2+x_6^2\pmod p.\]

\item Given a finite set $P$ of prime numbers, prove that there exists
  a positive integer $x$ which is representable in the form
  $x=a^p+b^p$ (with $a,b\in\mathbb N$) for each $p\in P$, but not
  representable in that form for any $p\notin P$.

\item Let $p>2$ be a prime and let $a,b,c,d$ be integers not divisible
  by $p$, such
  that \[\left\{\frac{ra}p\right\}+\left\{\frac{rb}p\right\}+\left\{\frac{rc}p\right\}+\left\{\frac{rd}p\right\}
  = 2\] for any integer $r$ not divisible by $p$.  Prove that at least
  two of the numbers $a+b$, $a+c$, $a+d$, $b+c$, $b+d$, $c+d$ are
  divisible by $p$.  Here, for real numbers $x$, $\{x\} = x-\lfloor
  x\rfloor$ denotes the fractional part of $x$.
\item Let $m$ be a given positive integer. \begin{enumerate} \item
  Prove that there exists an integer $N_1$ such that for every prime $p$
  greater than $N_1$, there are $m$ consecutive positive integers each
  of which is congruent to the square of an integer modulo $p$.
\item Prove that there exists an integer $N_2$ such that for every
  prime $p$ greater than $N_2$, there are $m$ consecutive positive
  integers each of which is not congruent to a square of an integer
  modulo $p$.
\end{enumerate}
\item Let $f,g:\mathbb N\rightarrow\mathbb N$ be functions with the
  properties: \begin{enumerate} \item $g$ is surjective; \item
    $2f(n)^2=n^2+g(n)^2$ for all positive integers $n$; \item
    $|f(n)-n|\leq 2004\sqrt n$ for all $n$. \end{enumerate} Prove that
  there are infinitely many $n\in\mathbb N$ with $f(n)=n$.
\item Find all positive integers $n>1$ for which there exists a unique
  integer $a$ with $0<a\leq n!$ such that $a^n+1$ is divisible by
  $n!$.
\item Let $p$ be a prime number.  Prove that there exists a prime
  number $q$ such that for every integer $n$, the number $n^p-p$ is
  not divisible by $q$.
\item Let $f$ be a polynomial with integer coefficients, and let $p$
  be a prime such that $f(0)=0$, $f(1)=1$, and $f(k)$ is congruent to
  either 0 or 1 modulo $p$ for all positive integers $k$.  Show that
  the degree of $f$ is at least $p-1$.
\item Find all integer solutions of the equation \[\frac{x^7-1}{x-1}=y^5-1.\]
\item Find all ordered triples of primes $(p,q,r)$ such
  that \[p|q^r+1,\quad q|r^p+1,\quad r|p^q+1.\]

\end{enumerate}

\end{document}
