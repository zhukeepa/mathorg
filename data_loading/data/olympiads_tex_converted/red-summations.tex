\documentclass[12pt]{article}

\usepackage{maria}

\title{Summations \\ \small{Red Group, MOP 2013}}

\author{Maria Monks}
\date{June 6, 2013}


\begin{document}

\maketitle{}

\section*{Some summation techniques}

\begin{itemize}
\item \textbf{Telescoping sums:}
Suppose we want to evaluate the sum $a_1+a_2+\cdots+a_n$, and we can write each $a_i$ as $b_i-b_{i+1}$ for some collection of numbers $b_i$.  Then all the $b_i$'s cancel except for $b_1$ and $b_{n+1}$, and we are left with $b_1-b_{n+1}$.

For instance, we can evaluate the sum $1+3+5+\cdots+(2k-1)$ by writing the $i$th odd number as $i^2-(i-1)^2$, obtaining the sum: $$(1^2-0^2)+(2^2-1^2)+(3^2-2^2)+\cdots+(k^2-(k-1)^2)=k^2-0^2=k^2.$$

%Telescoping products, such as $\frac{1}{2}\cdot \frac{2}{3} \cdot \frac{3}{4} \cdot\cdots\cdot \frac{n-1}{n}$, are useful in the same manner.

\item \textbf{Induction:} For finite sums, it's often possible to just guess the answer and then prove your guess by induction.

\item \textbf{Solve for the sum:} Let $S$ be the sum you wish to find, manipulate $S$ in order to simplify the sum, and finally solve for $S$.  For instance, to evaluate the infinite sum $$S=\sum_{j=0}^\infty\frac{j^2}{2^j},$$ we might consider multiplying by $2$ to shift the indices.  We have  $$2(S-1)=\sum_{j=1}^\infty \frac{j^2}{2^{j-1}}=\sum_{j=0}^\infty \frac{(j+1)^2}{2^j}$$  Subtracting $S$ from this, we find $2(S-1)-S=\sum_{j=0}^{\infty} \frac{2j+1}{2^{j}}$.  So far, we have reduced the $j^2$ to a $2j+1$ in the numerator.  Can you finish it from here?

\item \textbf{Generating functions:}  It can be useful to consider the sum as a coefficient in a generating function, especially in the following circumstances:
\begin{itemize}
\item \textbf{Partial sums:} Say we want to find a formula for $1^2+2^2+\cdots+n^2$.  First consider the generating function $S(x)=\sum_{i=0}^\infty n^2x^n$.  Now, if we divide by $1-x$, we get the partial sums as the coefficients of the new generating function: $$\frac{S(x)}{(1-x)}=\sum_{i=0}^\infty (1^2+2^2+\cdots+n^2)x^n.$$  So, we just need to find the coefficient of $x^n$ in $\frac{S(x)}{(1-x)}$, which equals $\frac{x^2+x^4}{(1-x)^4}$ (why?).  We expand $(1-x)^{-4}$ using the \textit{extended binomial theorem}:

\begin{eqnarray*}
\frac{1}{(1-x)^4} &=& \sum_{n=0}^\infty (-1)^n\binom{-4}{n}x^n \\
&=&\sum (-1)^n\frac{(-4)(-5)(-6)\cdots(-4-n+1)}{n!}x^n \\
     &=& \sum \frac{4\cdot 5\cdot \cdots\cdot (n+3)}{n!} x^n \\
     &=& \sum \frac{n(n+1)(n+2)}{6}x^n
\end{eqnarray*}

Multiplying this by $x+x^2$, the coefficient of $n$ in the result is $$\frac{(n-1)n(n+1)}{6}+\frac{n(n+1)(n+2)}{6}=\frac{n(n+1)(2n+1)}{6},$$ as desired.

\item \textbf{Convolutions:}  Sums of the form $$a_0b_n+a_1b_{n-1}+\cdots+a_nb_0$$ are called \textit{convolutions}, and they should alert your generating function radar.  Setting $F(x)=\sum a_nx^n$ and $G(x)=\sum b_nx^n$, this convolution is the coefficient of $x^n$ in $F(x)\cdot G(x)$.

Can you use this idea to compute $\sum_{i=0}^n i(n-i)$?

\end{itemize}

\end{itemize}

\section*{Problems}

\begin{enumerate}


\item Evaluate the sum $\sum_{i=0}^n i(n-i)$ using several different summation techniques.  Be creative!
%The answer turns out to be n(n^2-1)/6.  This is a good one.

\item Evaluate the sum $\sum_{k=1}^n \frac{k}{(k+1)!}$.
%Easily telescopes, or also by induction it turns out to be $1-\frac{1}{(n+1)!}$

\item Evaluate the sum $\sum_{k=1}^n k\cdot k!$.

\item (Brazilian Math Olympiad 2002.)  For a nonempty subset $A$ of $\{1,2,\ldots,n\}$ define $f(A)$ as the largest element of $A$ minus the smallest element of $A$.  Find $\sum f(A)$ where the sum is taken over all nonempty subsets $A$ of $\{1,2,\ldots,n\}$.
%Turns out to be (n-3)*2^n+n+3.  Do this by summing the values of all the min and max values separately: each i is the max for 2^{i-1} subsets, etc.

\item Evaluate the sum $\sum_{k=1}^n (-1)^{\frac{k(k+1)}{2}} k$.
% The answer turns out to be 0 when n is 3 mod 4, and the others are easily determined.  (You can prove the former by induction.)

\item (IMO Longlist 1978.)  Evaluate the sum $$1\cdot 2\cdot 3+2\cdot 3\cdot 4+\cdots+97\cdot 98\cdot 99.$$  (Bonus: What happens if we divide by $3!$ and think of the terms as binomial coefficients?)
%This turns out to be $97*98*99*100/4$, because of the hockey stick identity.  You can also do this by making each term $i(i-1)(i+1)$ and summing $i^3-i$, but this doesn't turn out as nicely in the formula in the end.

\item Find a closed formula for $\frac{1}{1\cdot 2 \cdot 3}+\frac{1}{2\cdot 3 \cdot 4}+\cdots+\frac{1}{n\cdot (n+1) \cdot (n+2)}.$
%%1/4 of n(n+3)/(n+1)(n+2), nicely enough.  Either: 1/1*(1/2-1/3)+1/2*(1/3-1/4)+... and it reduces to 2 products case, or: telescope by making it 1/(2n(n+1))-1/(2(n-1)n).  

\item (Art and Craft of Problem Solving.)  A 2-inch elastic band is fastened to the wall at one end, and there's a bug at the other end.  Every minute (beginning at time 0), the band is instantaneously and uniformly stretched by 1 inch, and then the bug walks 1 inch toward the fastened end.  Will the bug ever reach the wall?
%Yep!  It's a harmonic series thing.  Just record the percent along the band that the bug is at at each step.  The ratio starts at 1 and decreases by 1/3, 1/4, 1/5, ... so eventually the ratio will be less than 0, at which point he has reached the wall.

\item In all of the following, let $F_n$ be the $n$th Fibonacci number, where $F_0=0,F_1=1$, and $F_{n+2}=F_n+F_{n+1}$ for all $n\ge 0$.
\begin{enumerate}
\item Evaluate $\sum_{n=0}^\infty \frac{F_n}{F_{n+1}\cdot F_{n+2}}$.
%This telescopes because each term is equal to $1/F_{n+1}-1/F_{n+2}$.  We are left with $1/F_1=1$.
\item Evaluate $\sum_{n=1}^\infty \frac{F_{n+1}}{F_n\cdot F_{n+2}}$.
%Also telescopes, but now we have a jumping thing so it's $1/1+1/1=2$.
\item Evaluate $\sum_{n=1}^\infty \frac{1}{F_n\cdot F_{n+2}}$.
%Telescope this guy by showing that the nth term equals $1/(F_n*F_{n+1})-1/(F_{n+1}*F_{n+2})$.
\item Show that $\sum_{k=0}^n F_k=F_{n+2}-1$.
%Induction works just fine here
\item Show that $\sum_{k=0}^n F_k^2=F_n\cdot F_{n+1}$.
%Also induction.
\end{enumerate}

\item (Canada 1989.)  Given the numbers $1,2,2^2,\ldots,2^{n-1}$, for a specific permutation $\sigma=x_1,x_2,\ldots,x_n$ of these numbers we define $$S_i(\sigma)=x_1+x_2+\cdots+x_i$$ for each $i=1,2,\ldots,n$.  Define $Q(\sigma)=S_1(\sigma)S_2(\sigma)\cdots S_n(\sigma)$.  Evaluate $\sum 1/Q(\sigma)$ where the sum is taken over all possible permutations.
%The answer is 1/2^{n choose 2}.  Basically, you can show by induction that you can group them according to the last factors in the denominators.  You can always factor out a 1/S_n=1/(1+2+...+2^{n-1}), and then to sum what is left, group it by the next largest term, namely the $S_{n-1}(\sigma)$'s.  Use strong induction to show that these sum corresponding to a fixed $S_{n-1}(\sigma)$ turn out to be one over the product of the terms in the sum $S_{n-1}(\sigma)$.  Then in the end we add up all these sums, weighted by 1/S_n, and we get S_n/((product of terms in S_n)*S_n)=1/(1*2*4*...*2^{n-1})=1/2^{n choose 2}.

\item (Ukraine.) Show that $$\frac{1}{\sqrt{1}+\sqrt{3}}+\frac{1}{\sqrt{5}+\sqrt{7}}+\cdots+\frac{1}{\sqrt{9997}+\sqrt{9999}}>24.$$
%Add in the terms $1/(sqrt(3)+sqrt(5))$ etc, which at most doubles the LHS, and then rationalize denominators.  The sum now telescopes, and you can check that the resulting term is greater than 48.

\item (IMO 1996.)  Show that $$\frac{1}{\sin 2x}+\frac{1}{\sin 4x}+\cdots+\frac{1}{\sin 2^nx}=\cot x-\cot 2^nx.$$
%It turns out that 1/sin(2x)=cot(x)-cot(2x) and so the sum telescopes.

\item (Math Olympiad Challenges.) Prove the identity $$\sum_{k=1}^n \tan^{-1}\frac{1}{2k^2}=\tan^{-1}\frac{n}{n+1}$$
%By inverse of tan subtraction formula, show that the kth term is $\tan^{-1}(2k+1)-\tan^{-1}(2k-1)$ and the sum now telescopes.  Then use tan subtraction again at the end.
\end{enumerate}


\end{document}